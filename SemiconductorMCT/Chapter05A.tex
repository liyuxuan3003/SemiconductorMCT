\section{荧光光谱的基本概念}

\subsection{发光的基本概念}
\uwave{发光}(Luminescence)是指材料由于热辐射、光激发、电激发等原因,而产生电磁辐射光发射的一种现象。\uwave{光致发光}(Photoluminescence, PL)是指由于光激发而引起的材料的光发射。

发光可以分为两种类型,材料受光激发后
\begin{itemize}
    \item 由激发单重态跃迁回基态所发出的光辐射,称为荧光。
    \item 由激发三重态跃迁回基态所发出的光辐射,称为磷光
\end{itemize}
磷光行为自旋禁阻,发光强度很弱,因此材料发光通常为荧光,故发光光谱也称为荧光光谱。

光致荧光可以看作光吸收的逆过程,材料吸收一定波长的光,激发电子从低能级跃迁到高能级
\begin{itemize}
    \item 对于无机半导体晶体而言,体现为电子从价带到导带的激发。
    \item 对于原子或分子体系而言,体现为基态到激发态的跃迁。
\end{itemize}
高能级的电子是不稳定的,经过短暂的时间,电子又会跃迁到原来的低能量的能级,其中部分的能量以光的形式释放出来,即辐射跃迁,若能量以其他形式释放,则称为非辐射跃迁。

\begin{itemize}
    \item 组态自旋允许跃迁(激发态和基态具有相同的自旋多重度),荧光(Fluorescence)。
    \item 组态自旋禁阻跃迁(激发态和基态具有不同的自旋多重度),磷光(Phosphorescence)。
\end{itemize}

组态自旋禁阻跃迁(磷光)相较于组态自旋允许跃迁(荧光),只能通过非常弱的自旋轨道耦合方式发生,辐射跃迁速率更低,辐射跃迁寿命更长(磷光$\SIrange{1}{10}{ms}$,荧光$\SIrange{0.1}{10}{ns}$)。

\subsection{无机半导体材料的发光}

无机半导体材料的发光可以分为两种类型
\begin{itemize}
    \item 本征发光,由于带间辐射产生。
    \item 非本征发光,由于杂质和缺陷等产生的辐射跃迁引起。
\end{itemize}

\subsection{激发态的多重度}
激发态的多重度$M=2s+1$,其中$s$为电子自旋量子数的代数和
\begin{itemize}
    \item 当全部轨道里的电子都是自旋配对的,$s=0$,$M=1$,即分子体系处于单重态S。
    \item 当全部轨道有两个自旋不配对的电子,$s=1$,$M=3$,即分子体系处于三重态T。
\end{itemize}

\subsection{激发态的辐射与非辐射跃迁}
激发态的非辐射跃迁
\begin{itemize}
    \item 振动弛豫:激发态分子由于分子间的碰撞,由同一电子能级中较高振动能级转至较低振动能级的过程。振动弛豫的速率较快,因而激发态常常首先发生振动弛豫。
    \item 系间窜跃:激发态分子的电子自旋发生倒转而使分子的多重态发生变化的过程。
    \item 外转换:激发态分子与周围分子相互作用和能量转换而激发态减弱甚至消失的过程。
    \item 内转换:激发态分子内,两个相同多重态的电子能级间,电子由高能级跃迁至低能级。
\end{itemize}
激发态的辐射跃迁
\begin{itemize}
    \item 荧光(Fluorescence):组态自旋允许跃迁产生的光。
    \item 磷光(Phosphorescence):组态自旋禁阻跃迁产生的光,磷光比荧光红移。
\end{itemize}

\subsection{激发光谱与发射光谱}
激发光谱:将荧光的最大波长选择为测量波长,改变激发光的波长,测量发光强度的变化。

发射光谱:将激发光的波长和强度固定,改变不同发射波长,测定发光强度的变化。

激发光谱和发射光谱间
\begin{itemize}
    \item 存在斯托克斯位移,发射光谱相较激发光谱有红移,因为振动弛豫消耗了能量。
    \item 存在镜像规则,发射光谱与它的激发光谱成镜像对称关系。
    \item 发射光谱的形状与激发波长无关。
\end{itemize}


\subsection{荧光和磷光的量子效率}
荧光的量子效率
\begin{Equation}
    \phi_F=\frac{k_F}{k_F+\Sum[i=1][n]k_i}
\end{Equation}
磷光的量子效率
\begin{Equation}
    \phi_P=\frac{k_P\phi_{ST}}{k_P+\Sum[i=1][n]k_i}
\end{Equation}
其中,$k_F,k_P$主要取决于发光物质的分子结构,$\phi_{ST}$为系间窜跃效率,$k_i$为非辐射跃迁速率。

\subsection{影响荧光强弱的因素}
强荧光的有机化合物具有以下特征
\begin{itemize}
    \item 具有大的共轭$\pi$键的结构。
    \item 具有刚性的平面结构。
    \item 最低激发单重态到基态的跃迁为$\pi^{*}\to\pi$型。
\end{itemize}

\subsubsection{跃迁的类型}
对于有机荧光物质
\begin{itemize}
    \item $\pi^{*}\to n$跃迁的辐射跃迁速率低,且$\phi_{ST}$大。
    \item $\pi^{*}\to\pi$跃迁的辐射跃迁速率高,且$\phi_{ST}$小,是有机物荧光的主要跃迁类型。
\end{itemize}

\subsubsection{共轭效应}
共轭双键结构有利于发光,共轭度越大,荧光效率就越大,且向长波方向移动。

\subsubsection{刚性平面结构}
刚性和平面性增加,荧光发射增强,这种构型可以减少分子振动和系间窜跃的可能性。

\subsubsection{取代基效应}
取代基的影响和类型有关
\begin{itemize}
    \item 给电子基团,使荧光增强(增强$\pi$电子的共轭程度),例如\cx{-NH2},\cx{-NHR},\cx{-OH},\cx{-OR}
    \item 吸电子基团,使荧光减弱(引入了$\pi^{*}\to n$跃迁),例如\cx{-COOH},\cx{-C=O},\cx{-NO2}
\end{itemize}

\subsubsection{重原子效应}
引入\cx{-Cl}、\cx{-Br}、\cx{-I}等重原子会增加系间窜跃速率,减弱荧光,增强磷光。

\subsubsection{环境的影响}
\begin{itemize}
    \item 溶剂:溶剂的极性越大,荧光波长红移。
    \item 温度:温度增加,荧光效率和荧光强度降低。
    \item 酸碱:弱酸弱碱对荧光强度和荧光光谱产生有很大影响,因为酸碱性会影响分子构型。
\end{itemize}


\subsubsection{荧光猝灭}
荧光物质与溶剂分子或其他溶质分子相互作用,引起荧光强度下降或消失,称为\uwave{荧光猝灭}
\begin{itemize}
    \item 浓度猝灭:物质浓度大于$\SI{1}{g.L^{-1}}$时,常发生荧光的自猝灭。
    \item 碰撞猝灭:荧光分子与猝灭剂碰撞后,以无辐射跃迁返回基态。
    \item 能量转移猝灭:荧光分子与猝灭剂发生电子转移反应或能量传递。
\end{itemize}
