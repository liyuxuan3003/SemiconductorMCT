\section{紫外--可见吸收光谱的原理}
有机物的紫外--可见吸收光谱是其分子外层电子跃迁的结果,电子可以分为三类
\begin{itemize}
    \item $\sigma$电子,即成$\sigma$键的电子(单键),对应成键轨道$\sigma$和反键轨道$\sigma^{*}$。
    \item $\pi$电子,即成$\pi$键的电子(双键),对应成键轨道$\pi$和反键轨道$\pi^{*}$。
    \item $n$电子,即未成键的电子,对应非键轨道$n$。
\end{itemize}

\xref{fig:电子能级跃迁示意图}列出了各能级间的相对位置关系和可能的跃迁
\begin{Figure}[电子能级跃迁示意图]
    \includegraphics{build/Chapter04A_01.fig.pdf}
\end{Figure}

\xref{tab:电子能级跃迁}整理了四种跃迁过程的特点
\begin{Tablex}[电子能级跃迁]{lllXl}
<类型&能量&区段&波长$\lambda_{\max}$&备注\\>
$\sigma\to\sigma^{*}$跃迁&能量最大&远紫外&甲烷$\SI{125}{nm}$、乙烷$\SI{135}{nm}$\\
$n\to\sigma^{*}$跃迁&能量较大&远紫外&甲醇$\SI{183}{nm}$、一氯甲烷$\SI{173}{nm}$\\
$\pi\to\pi^{*}$跃迁&能量较小&近紫外&乙烯$\SI{162}{nm}$&$\varepsilon_{\max}=\SI{1e4}{L.mol^{-1}.cm^{-1}}$\\
$n\to\pi^{*}$跃迁&能量最小&近紫外&丙酮$\SI{275}{nm}$&$\varepsilon_{\max}=\SI{1e2}{L.mol^{-1}.cm^{-1}}$\\
\end{Tablex}

由于$\pi$键是有机半导体分子的特征结构,因此,有机半导体材料的紫外--可见光谱分析也多以涉及$\pi$电子轨道的$\pi\to\pi^{*}$跃迁和$n\to\pi^{*}$跃迁为主,前者强度可以达到后者的$100$倍以上。

\subsection{生色与助色}
紫外可见光谱仪可以分析的波长范围($\SIrange{200}{780}{nm}$)主要由$\pi\to\pi^{*}$跃迁和$n\to\pi^{*}$跃迁产生。这两种跃迁要求分子具有含$\pi$键的不饱和基团,因此这些含有$\pi$键的不饱和基团被称为\uwave{生色团}。简单的生色团由双键和叁键体系组成。同时,有一些含有$n$电子的基团,它们本身没有生色功能,即不能吸收大于$\SI{200}{nm}$的光,但是,它们与生色团链接时,就会发生$n$--$\pi$共轭作用,增强生色团的生色能力,因此这样的基团称为\uwave{助色团}。以下是常见的生色团和助色团
\begin{itemize}
    \item 生色团:\cx{-C=C-}、\cx{-C#C-}、\cx{-C=O}、\cx{-C#N}等
    \item 助色团:\cx{-OH}、\cx{-NH2}、\cx{-X}等
\end{itemize}

助色团对生色团的影响是:吸收波长向长波方向移动,吸收强度增加。

\subsection{红移与蓝移}
紫外可见光谱中,有机化合物的吸收谱常常因引入取代基或改变溶剂,使$\lambda_{\max}$和$\varepsilon_{\max}$变化
\begin{itemize}
    \item 最大吸收波长$\lambda_{\max}$,增加称为红移,减小称为蓝移。
    \item 最大吸收强度$\varepsilon_{\max}$,增加称为增色,减小称为减色。
\end{itemize}
这里的吸收强度$\varepsilon_{\max}$其实就是摩尔消光系数,即$A=\varepsilon bc$中的$\varepsilon$。

\subsection{摩尔消光系数}
材料的摩尔吸光系数$\varepsilon$不随浓度$c$和光程长度$b$的改变而改变,仅与吸收材料自身的性质有关。同一物质在不同波长下的$\varepsilon$不同,最强吸收峰对应的$\lambda$和$\varepsilon_{\max}$分别记为$\lambda_{\max}$和$\varepsilon_{\max}$。\goodbreak

吸收光谱中,我们会依据以下参量进行定性分析
\begin{itemize}
    \item 吸收峰$\lambda_{\max}$
    \item 吸收谷$\lambda_{\min}$
    \item 肩峰$\lambda_{sh}$
    \item 最大摩尔消光系数$\varepsilon_{\max}$
\end{itemize}

\subsection{影响紫外光谱的因素}
有诸多因素会影响紫外光谱
\begin{itemize}
    \item 助色基团:助色基团的引入,红移,增色。
    \item 共轭效应:共轭的效应越强,红移,增色(乙烯,丁二烯,己三烯,辛四烯)。
    \item 溶剂的影响
    \begin{itemize}
        \item 溶剂对吸收峰位置的影响:$\pi\to\pi^{*}$跃迁,溶剂极性增大,吸收峰红移。
        \item 溶剂对吸收峰位置的影响:$n\to\pi^{*}$跃迁,溶剂极性增大,吸收峰蓝移。
        \item 溶剂对吸收峰结构的影响:极性增大,吸收峰的振动精细结构消失。
    \end{itemize}
\end{itemize}
因此,通常选用低极性溶剂,且溶剂在样品的吸收光谱区无明显吸收。