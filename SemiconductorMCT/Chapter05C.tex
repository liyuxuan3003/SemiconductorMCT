\section{荧光光谱的应用}

\subsection{定量分析}
根据荧光发生的机理可知,溶液的荧光强度$F$和该溶液的吸收光强度$I_a$以及荧光效率$\phi_F$成正比
\begin{Equation}
    F=\phi_F\cdot I_a
\end{Equation}
根据朗伯比尔定律,溶液吸收的光量为
\begin{Equation}
    I_a=I_0-I_t=I_0(1-10^{-\varepsilon bc})
\end{Equation}
其中,$I_0$为入射光强度,$I_t$为透射光强度,$\varepsilon$为摩尔吸光系数。

结合上两式可以得到
\begin{Equation}
    F=\phi_FI_0(1-\e^{-2.3\varepsilon bc})
\end{Equation}
对于稀溶液,当$\varepsilon bc$小于$0.02$时,上式可以近似为
\begin{Equation}
    F=2.3\phi_FI_0\varepsilon bc
\end{Equation}
注意,低浓度时成线性关系,高浓度时由于自猝灭和自吸收等原因不成线性关系。

\subsubsection{直接测定法}
直接测定法适用于本身能发荧光的物质。

直接测定法:配置一系列已知浓度的标准溶液,并测定它们的荧光强度,以荧光强度对标准溶液浓度绘制标准曲线。然后将待测试样溶液的荧光强度代入标准曲线,得到待测组分浓度。

\subsubsection{间接测定法}
间接测定法适用于本身不能发荧光的物质。
\begin{itemize}
    \item 荧光衍生法:待测物质与荧光试剂通过化学反应形成能发荧光的衍生物。
    \item 荧光猝灭法:待测物质本身不发光,但能使得某种有荧光的物质发生荧光猝灭。
\end{itemize}

\subsection{定性分析}
比较法:将待测物质的荧光发射光谱与预期化合物的荧光发射光谱进行比较,一致则为该物质。