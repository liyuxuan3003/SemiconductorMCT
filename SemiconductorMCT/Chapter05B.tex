\section{荧光光谱仪的光源}

荧光光谱仪由几个部分构成:光源、样品室、单色器、检测器。

\subsection{光源}
荧光光谱仪的光源
\begin{itemize}
    \item 发射强度足够且稳定的连续光
    \item 光辐射强度随波长变化小
    \item 有足够长的使用寿命
\end{itemize}
荧光光谱仪常用的光源是氙灯光源,发光波长范围宽$\SIrange{200}{1000}{nm}$,发射光强度大。

\subsection{样品室}
溶液:采用低荧光吸收的石英材质的方形或长方形池体。

固体:采用石英片衬底。

\subsection{单色器}
荧光光谱仪具有两个单色器
\begin{itemize}
    \item 第一个单色器置于光源和试样池之间,用于选择所需的激发波长。
    \item 第二个单色器置于试样池与检测器间,用于分离所需检测的波长。
\end{itemize}

\subsection{检测器}
荧光的强度较弱,需要高灵敏度检测器,采用光电管或光电倍增管。