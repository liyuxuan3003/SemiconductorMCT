\section{质谱法的构造}
质谱法构造组成如下
\begin{itemize}
    \item 进样系统
    \begin{enumerate}
        \item 间歇进样
        \item 直接进样
        \item 色谱进样
    \end{enumerate}
    \item 离子源
    \begin{enumerate}
        \item 电子轰击
        \item 化学电离
        \item 场致电离
        \item 激光
    \end{enumerate}
    \item 质量分析器
    \begin{enumerate}
        \item 单聚焦
        \item 双聚焦
        \item 飞行时间
        \item 四极杆
    \end{enumerate}
    \item 检测器
\end{itemize}

\subsection{高真空要求}
质谱仪的离子源、质量分析器及检测器必须处于高真空状态
\begin{itemize}
    \item 质谱仪的离子源的真空度应达到$\SIrange{e-3}{e-5}{Pa}$
    \item 质量分析器的真空度应达到$\SI{e-6}{Pa}$
\end{itemize}
若真空度低,则
\begin{itemize}
    \item 离子源的灯丝会被大量氧烧坏。
    \item 离子源中的电子束的正常调节会被干扰。
    \item 本底增高,引发额外的离子--分子翻译,改变裂解模型,使质谱解释复杂化。
    \item 用作加速离子的几千伏高压会引发放电。
\end{itemize}

\subsection{质谱仪的进样系统}
\uwave{进样系统}的作用是高效重复的将样品引入到离子源中,并且,该过程中,不能使真空度降低。

\subsubsection{间歇式进样系统}
\uwave{间歇式进样系统},适用于低沸点易挥发的液体或固体,或气体样品。

间歇式进样系统通过可拆卸式的试样管将少量$\SIrange{10}{100}{ug}$固体或液体试样引入试样储存器中,由于进样系统的低压强和储存器的加热装置,试样将保持气态。由于进样系统的压强比离子源的压强要大,样品离子可以通过分子漏隙以分子流的形式渗透进入高真空的离子源。

\subsubsection{直接探针进样系统}
\uwave{直接探针进样系统},适用于高沸点的液体或固体。直接探针进样系统通过探针(Probe)杆直接进样,调节加热温度,使试样转化为蒸汽。此方法可以将$\SI{1}{ug}$量级甚至更少的试样送入电离室。探针杆中试样的温度,可以冷却至约$\SI{-100}{\dc}$,同时,可以在几秒内加热至约$\SI{300}{\dc}$。

\subsubsection{色谱进样}
\uwave{色谱进样},利用气相和液相色谱的分离能力,与质谱仪连用,进行多组分复杂混合物的分析。

\subsection{质谱仪的离子源}
\uwave{离子源}的功能是将进样系统引入的气态样品分子转化为离子,由于离子化所需的能量随分子的不同差异很大,因此,对于不同的分子,应选择不同的离子化方法,具体而言
\begin{itemize}
    \item 给样品较大能量的电离方法,称为\uwave{硬电离方法}。
    \item 给样品较小能量的电离方法,称为\uwave{软电离方法}。
\end{itemize}
离子化的方法有很多
\begin{itemize}
    \item 电子轰击电离(Electron Impaction Ionization, EI)
    \item 化学电离(Chemical Ionization, CI)
    \item 场电离(Field Ionization, FI)
    \item 场解吸(Field Desorption, FD)
    \item 快原子轰击(Fast Atom Bombardment, FAB)
    \item 基质辅助激光解析电离(Matrix-Assisted Laser Desorption Ionization, MALDI)
    \item 电喷雾电离(Electrosparay Ionization, ESI)
    \item 大气压化学电离(Atmospheric Pressure Chemical Ionization, APCI)
\end{itemize}

\subsubsection{电子轰击电离EI}
\uwave{电子轰击电离}是指,使用高能电子束从试样中撞出一个电子而产生电子,即
\begin{Chemeq}
    M + e- -> M+ + 2e-
\end{Chemeq}
其中,\cx{M}为待测分子,\cx{M+}为分子离子,当电子轰击源有足够的能量,通常约$\SI{70}{eV}$时,有机物的分子可能不仅失去一个电子形成分子离子,而且还有可能进一步发生键的断裂,形成大量的各种低质量束的碎片离子和中性自由基团,这些碎片离子也可以辅助结构鉴定。

具体而言,试想待测分子\cx{M}为\cx{ABC}
\begin{itemize}
    \item \uwave{分子离子}:分子失去一个电子形成的离子,即\cx{ABC+}
    \item \uwave{碎片离子}:分子碎裂后形成的离子,比如可能有\cx{A+},\cx{B+},\cx{C+},\cx{AB+},\cx{AC+}等
\end{itemize}
% 电子轰击电离的优点
% \begin{itemize}
%     \item 电子轰击电离的仪器结构简单,稳定,电离效率高,易于实现
%     \item 电子轰击电离的质谱图再现性好,便于计算机检索及比较
%     \item 电子轰击电离产生的碎片离子多,可提供较多的分子
% \end{itemize}

\subsubsection{化学电离CI}
\uwave{化学电离}是指,使离子室内的的反应气(甲烷、异丁烷、氨等,反应气压强$\SIrange{10}{100}{Pa}$,反应气是样品的$\SIrange{1e3}{1e5}{}$倍)被电子($\SIrange{100}{240}{eV}$)轰击先产生离子,离子再与试样分子碰撞反应。简而言之,EI是用电子轰击样品,CI是用化学反应产生的离子轰击样品。

化学电离CI产生的碎片离子较少,电子轰击电离EI产生的碎片离子较多。

\subsubsection{场电离FI}
\uwave{场电离}是一种软电离技术。当样品蒸汽邻近或接触到带高正电位的金属针时,由于高曲率的针端产生很强的电位梯度(即尖端放电,曲率越大电场越强),样品分子可被电离。

场电离,优点是电离快速适合与气相色谱联机,缺点是要求样品气化,灵敏度低。

\subsubsection{场解吸FD}
\uwave{场解吸}的原理与场电离相同,但样品是被沉积在电极上。

场解吸,适用于难气化且热不稳定的样品。

场解吸FD相较场电离FI,准分子离子峰更强,谱图更简单。

\subsubsection{快原子轰击FAB}
\uwave{快原子轰击}也是一种广泛应用的软电离技术,快原子轰击的原理是,惰性气体\cx{Ar}或\cx{Xe}原子首先被电离并加速,使之具有高的动能,在原子枪内进行电荷交换反应
\begin{Chemeq}
    Ar+\te{(高动能的)} + Ar\te{热运动的} -> Ar\te{(高动能的)} + Ar+\te{(热运动的)}
\end{Chemeq}
快原子轰击的原理是:先产生高动能的\cx{Ar}或\cx{Xe}原子束,再轰击样品分子使其离子化。

快原子轰击FAB可完成连场解吸FD都有困难的,高极性,难气化的化合物的电离。

\subsubsection{基质辅助激光解析电离MALDI}
\uwave{基质辅助激光解析电离}再一个微小的区域内,在极短的时间间隔,激光可对靶物提供高的能量,对它进行极快的加热,可以避免热敏感的化合物加热分解。基质辅助激光解析电离的优点是,使一些较难电离的样品电离,且无明显碎裂,从而得到完整的被分析化合物分子的电离产物,基质辅助激光解析电离MALDI特别适用于与飞行时间质谱相配,即MALDI-TOF MS。

\subsubsection{电喷雾电离ESI}
\uwave{电喷雾电离}是从雾化器套管的毛细管喷出的带电液滴,液滴在运动中溶剂不断快速蒸发,液滴迅速变小,表面电荷密度不断增大。由于电荷间的排斥作用,就会排出溶剂分子,得到样品的准分子离子。电喷雾电离ESI是很软的电离方法,通常无碎片离子峰,只有整体离子峰。

\subsection{质谱仪的质量分析器}
\uwave{质量分析器}是质谱仪的核心,它的作用,是将离子源产生的离子,按质荷比分开。

不同类型的质量分析器构成不同类型的质谱仪(质量分析器是质谱仪的核心)。

不同类型的质量分析器的功能、应用范围、原理、实验方法均有不同
\begin{itemize}
    \item 磁质量分析器(Magnetic Sector Analyzer)
    \item 四级质量分析器(Quadrupole Mass Analyzer)
    \item 四级离子阱(Quadrpole Ion Trap)
    \item 飞行时间质谱计(Time of Flight, TOF)
\end{itemize}

\subsubsection{磁质量分析器}
\uwave{单聚焦}(Single Focusing)和\uwave{双聚焦}(Double Focusing)质量分析器。

单聚焦质量分析器:样品电离后,先经过一个高压电场$U$加速,然后进入质量分析器,在磁场$B$的作用,其运动轨道发生偏转改作圆周运动,其运动半径可以表示为
\begin{Equation}
    zeVB=\frac{mv^2}{R}\qquad zeU=\frac{1}{2}mv^2
\end{Equation}
即
\begin{Equation}
    R=\frac{\sqrt{2}}{B}\sqrt{\frac{m}{ze}U}
\end{Equation}
在一定的$B,U$下,不同$m/z$的离子其运动半径$R$不同。

双聚焦质量分析器:样品电离后,先经过静电分析器,再经过磁分析器。静电分析器是由两个同心圆板组成,两者间保持某一电位差$E$,离子在静电分析器中所需的向心力由电场力提供
\begin{Equation}
    zeE=\frac{mv^2}{R}
\end{Equation}
即允许一定速度的离子通过,能量聚焦。

双聚焦和单聚焦质谱仪体积大,随着仪器的小型化要求,需要更小的质量分析器。

\subsubsection{四级质量分析器}
\uwave{四级质量分析器}由四根平行的金属杆组成,被加速的离子束穿过对准四根极杆之间空间的准直小孔。通过在四极上加上直流电压$U$和射频电压$V\cos\omega t$,在极间形成一个射频场,离子进入射频场后,只有合适$m/z$的离子才会通过稳定的振荡进入检测器。当扫描$U$和$V$并保持$U/V$的比值恒定时,不同的质荷比$m/z$的离子依次通过,从而被仪器检测到。

\subsubsection{四级离子阱}
\uwave{四级离子阱}原理与四级质量分析器类似,由一环形电极两个端罩电极构成,通过电场或磁场将气相离子控制并储存一段时间的装置。特定的$m/z$的离子在阱内一定轨道上稳定旋转,改端电极电压,不同$m/z$离子飞出阱达到检测器。四级离子阱具有高灵敏度和高质量范围。

\subsubsection{飞行时间质谱计}
飞行时间质谱计用一个脉冲将离子源中的离子瞬间引出,经电压加速,它们具有相同的动能进入漂移管,质荷比$m/z$小的离子先到达检测器,质荷比$m/z$大的离子后到达检测器。