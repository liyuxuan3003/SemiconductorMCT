\section{质谱法的性能指标}
质谱仪的性能指标:质量范围、灵敏度、分辨率、质量准确度。

\subsection{质量范围}
\uwave{质量范围}:质谱仪能够进行分析的样品的相对原子质量的最大范围。

质谱范围通常采用原子质量单位(Unified Atomic Mass Unit, amu)进行度量
\begin{itemize}
    \item 磁质谱:$\SIrange{1e0}{1e4}{amu}$
    \item 四极杆质谱:$\SIrange{1e1}{1e3}{amu}$
    \item 离子阱质谱:$\SIrange{5e1}{2e3}{amu}$
    \item 飞行时间质谱:无上限。
\end{itemize}

\subsection{分辨率}
\uwave{分辨率}:分离质量数$M_1$及$M_2$的相邻质谱峰的能力。

分辨率通常定义为(其中$W_{0.05}$为峰高$5\%$处的峰宽)
\begin{Equation}
    R=\frac{M_1}{W_{0.05}}
\end{Equation}
不同质量分析器的分辨率
\begin{itemize}
    \item 单聚焦:$\SI{5e3}{}$
    \item 双聚焦:$\SI{1.6e5}{}$
    \item 四极杆:$\SI{2e3}{}$
\end{itemize}

\subsection{灵敏度}
\uwave{灵敏度}:对于一定样品,在一定的分辨率下,产生一定信噪比的分子离子峰所需的样品量。

灵敏度有三种表示方法
\begin{enumerate}
    \item 绝对灵敏度:仪器可检测到的最小样品量。
    \item 相对灵敏度:仪器可以同时检测的大小组分含量之比。
    \item 分析灵敏度:输入仪器的样品量与仪器产生的信号之比。
\end{enumerate}

\subsection{质量准确度}
\uwave{质量准确度}:即离子质量实测值$M$与理论值$M_0$的相对误差。