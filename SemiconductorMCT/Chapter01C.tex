\section{核磁共振碳谱}
碳谱和氢谱原理是类似的,但有很多不同,首先,碳的常见同位素是\cx{^{12}C},然而\cx{^{12}C}并不是磁性核,碳能参与核磁共振的同位素是\cx{^{13}C},然而\cx{^{13}C}的天然丰度只有$1.1\%$,占比非常小,因此,碳谱相较氢谱,灵敏度低,信噪比差。但是,碳谱也有很多氢谱所不具备的优点,包括
\begin{itemize}
    \item 碳谱可以掌握碳骨架的结构信息
    \item 碳谱的化学位移范围更大,可以超过\SI{200}{ppm}。
    \item 碳谱可以确定碳原子级数
    \item 碳谱的图谱也较为简单。
\end{itemize}
碳谱的另外一个问题是,碳氢的耦合常数非常大,这会使谱图无意义的复杂化。解决的办法被称为\uwave{宽带去耦}。即以相当宽的频带照射频率,使氢的自旋耦合作用被平均,解除碳氢耦合。