\section{红外吸收光谱的原理}

\subsection{分子的振动模式}
分子的振动主要可以分为两类:伸缩振动、弯曲振动。

\subsubsection{分子的伸缩振动}
\uwave{伸缩振动}:只改变键长,不改变键角,吸收频率相对在高波数区/高频(近红外)
\begin{itemize}
    \item 对称的伸缩振动,强吸收,波数$\SI{2853}{cm^{-1}}$
    \item 反称的伸缩振动,强吸收,波数$\SI{2853}{cm^{-1}}$
\end{itemize}

\subsubsection{分子的弯曲振动}
\uwave{弯曲振动}:只改变键角,不改变键长,吸收频率相对在低波数区/低频(中红外)
\begin{itemize}
    \item 面内弯曲振动,中吸收
    \begin{itemize}
        \item 面内摇动,波数$\SI{720}{cm^{-1}}$
        \item 剪切振动,波数$\SI{1468}{cm^{-1}}$
    \end{itemize}
    \item 面外弯曲振动,弱吸收
    \begin{itemize}
        \item 扭曲,波数$\SI{1250}{cm^{-1}}$
        \item 摇摆,波数$\SI{1303}{cm^{-1}}$
    \end{itemize}
\end{itemize}

\subsubsection{双原子分子的吸收频率}
从经典力学的观点,采用谐振子模型来研究双原子分子的振动,即化学键相当于无质量的弹簧,它连接两个刚性小球,两个刚性小球的质量分别相当于两个原子的质量,由此,频率为
\begin{Equation}
    \nu=\frac{1}{2\pi}\sqrt{\frac{k(m_1+m_2)}{m_1m_2}}=\frac{1}{2\pi}\sqrt{\frac{k}{\mu}}
\end{Equation}
容易由$\hat{\nu}=\nu/c$得到波数
\begin{Equation}
    \hat{\nu}=\frac{1}{2\pi c}\sqrt{\frac{k(m_1+m_2)}{m_1m_2}}=\frac{1}{2\pi c}\sqrt{\frac{k}{\mu}}
\end{Equation}
这里$\mu$称为原子$m_1,m_2$的折合质量
\begin{Equation}
    \mu=\frac{m_1m_2}{m_1+m_2}
\end{Equation}
这里可以得出以下结论
\begin{itemize}
    \item 化学键键能越强即$k$越大,原子折合质量越小$m$越小,吸收峰将出现在高波数区。
    \item 化学键键能越弱即$k$越小,原子折合质量越大$m$越大,吸收峰将出现在低波数区。
\end{itemize}

\subsubsection{红外光谱的产生条件}
分子的振动能级
\begin{Equation}
    E_v=\qty(V+\frac{1}{2})h\nu
\end{Equation}
这里,$V$为振动量子数,$\nu$为振动频率,而任意两个相邻能级的能量差即
\begin{Equation}
    \delt{E_v}=h\nu
\end{Equation}
这就表明,吸收红外光的能量恰等于振动跃迁所需的能量。

\subsection{红外光谱的峰数}
红外光谱的峰数取决于跃迁的形式
\begin{itemize}
    \item 由$V=0$跃迁至$V=1$,即基频峰。
    \item 由$V=0$跃迁至$V=2$,即二倍频峰。
    \item 由$V=0$跃迁至$V=3$,即三倍频峰。
\end{itemize}
倍频峰中,二倍频峰还比较强,三倍频峰及以上由于跃迁概率很小,一般都很弱,常常不能检测到,除此之外,还有合频峰(例如$V_1+V_2$)和差频峰(例如$V_1-V_2$),这些峰通常也很弱。

倍频峰、合频峰、差频峰,合称为泛频峰。

\subsection{红外光谱的峰强}
\uwave{朗伯--比尔定律}的基本内容是:光吸收过程,光的能量是按指数形式衰减的。

朗伯--比尔定律在这里就是称,透射比可以表示为
\begin{Equation}
    T=\frac{I}{I_0}=10^{-\varepsilon bc}
\end{Equation}
其中,$\varepsilon$为摩尔消光系数,$b$为光程,$c$为浓度。

吸光度定义为
\begin{Equation}
    A=-\log T=-\log\frac{I}{I_0}
\end{Equation}
吸光度因此可以表示为
\begin{Equation}
    A=\varepsilon bc
\end{Equation}\goodbreak
峰强与键的极性(键两端原子的电负性差值)有关,具体而言
\begin{itemize}
    \item 极性较强的基团,吸收强度较大,例如\cx{C=O}、\cx{C-X}等。
    \item 极性较弱的基团,吸收强度较小,例如\cx{C=C}、\cx{C-C}、\cx{N=N}等
\end{itemize}
红外光谱的吸收强度通常依\xref{tab:红外光谱的峰强}定性表示,以摩尔吸光系数$\varepsilon$划分
\begin{Tablex}[红外光谱的峰强]{XXXX}
<峰强等级&峰强记号&峰强记号含义&摩尔吸光系数的范围\\>
非常强峰&vs&very strong&\num{>1e2}\\
强峰&s&strong&\numrange{2e1}{1e2}\\
中强峰&m&medium&\numrange{1e1}{2e1}\\
弱峰&w&weak&\numrange{1e0}{1e1}\\
非常弱峰&vw&very weak&\num{<1e0}\\
\end{Tablex}

\subsection{官能团的特征频率}
化学键振动的特性性,用基团频率来表现
\begin{itemize}
    \item 特征性:同类型的基团具有特定的频率。
    \item 差异性:同类型的基团在不同的化学环境中频率略有差别。
\end{itemize}
中红外光谱的波数范围是$\SIrange{4e3}{4e2}{cm^{-1}}$,其特征频率如\xref{tab:官能团的特征频率}所示
\begin{Tablex}[官能团的特征频率]{llX}
<区域名称&波数范围($\si{cm^{-1}}$)&基团及振动形式\\>
氢键区&\numrange{4.0e3}{2.5e3}&\cx{O-H}、\cx{C-H}、\cx{N-H}等的伸缩振动\\
三键区\footnotemark&\numrange{2.5e3}{2.0e3}&\cx{C#C}、\cx{C#N}、\cx{N#N}、\cx{C=C=C}、\cx{N=C=O}等的伸缩振动\\
双键区&\numrange{2.0e3}{1.5e3}&\cx{C=O}、\cx{C=C}、\cx{C=N}、\cx{NO2}、\cx{C6H6}等的伸缩振动\\
单键区&\numrange{1.5e3}{4.0e2}&\cx{C-C}、\cx{C-O}、\cx{C-N}、\cx{C-X}等的伸缩振动,含氢基团的弯曲振动\\
\end{Tablex}\footnotetext{这里三键区其实包含了三键和累积双键两种情况。}

中红外光谱中,最有分析价值的基团频率在$\SIrange{4.0e3}{1.3e3}{cm^{-1}}$,这一区间也倍称为\uwave{特征频率区},改区内的峰是由伸缩振动产生的吸收带,比较系数,容易辨认,除此之外
\begin{itemize}
    \item 指纹区,是指$\SI{<1.6e3}{cm^{-1}}$的低频区,指纹区的特点是谱带密集且难以辨认,指纹区对整个分子结构十分敏感,常用于与标准谱图进行对比,以辨别官能团相似的物质。
    \item 倍频区,是指$\SI{>3.7e3}{cm^{-1}}$的高频区,倍频区的吸收峰是基本频率的倍频。
\end{itemize}\goodbreak
有很多因素可以影响官能团的吸收频率(差异性,化学环境对官能团频率的影响)
\begin{itemize}
    \item 内部因素:诱导效应、共轭效应、杂化效应、氢键效应
    \item 外部因素:溶剂的类型、物质的状态
\end{itemize}

\subsubsection{诱导效应}
诱导效应:羰基在连接电负性较强的基团时,吸收峰向高频移动(蓝移)。
\begin{Figure}[诱导效应]
    \begin{FigureSub}[\SI{1812}{cm^{-1}};溴羰基]
        \includegraphics{build/Chapter03B_01.fig.pdf}
    \end{FigureSub}
    \hspace{1cm}
    \begin{FigureSub}[\SI{1815}{cm^{-1}};氯羰基]
        \includegraphics{build/Chapter03B_02.fig.pdf}
    \end{FigureSub}
    \hspace{1cm}
    \begin{FigureSub}[\SI{1869}{cm^{-1}};氟羰基]
        \includegraphics{build/Chapter03B_03.fig.pdf}
    \end{FigureSub}
\end{Figure}

\subsubsection{共轭效应}
共轭效应:羰基与别的双键共轭,吸收峰向低频移动(红移)。
\begin{Figure}[共轭效应]
    \begin{FigureSub}[\SI{1715}{cm^{-1}};无苯]
        \includegraphics{build/Chapter03B_04.fig.pdf}
    \end{FigureSub}\\ \vspace{0.5cm}
    \begin{FigureSub}[\SI{1685}{cm^{-1}};1苯]
        \includegraphics{build/Chapter03B_05.fig.pdf}
    \end{FigureSub}\hspace{1.5cm}
    \begin{FigureSub}[\SI{1660}{cm^{-1}};2苯]
        \includegraphics{build/Chapter03B_06.fig.pdf}
    \end{FigureSub}
\end{Figure}

\subsubsection{杂化效应}
杂化效应:s轨道越多,键能越大,键长越短,吸收在高频区。
\begin{Figure}[杂化效应]
    \begin{FigureSub}[\SI{1430}{cm^{-1}};单键]
        \chemfig{-C-C-}
    \end{FigureSub}
    \hspace{1cm}
    \begin{FigureSub}[\SI{1660}{cm^{-1}};双键]
        \chemfig{-C=C-}
    \end{FigureSub}
    \hspace{1cm}
    \begin{FigureSub}[\SI{2220}{cm^{-1}};三键]
        \chemfig{-C~C-}
    \end{FigureSub}
\end{Figure}

\subsubsection{氢键效应}
氢键效应:氢键对峰位和峰强产生明显影响,吸收峰向低频区移动。

氢键效应的原理是,氢键的形成使原化学键的键力常数降低,从而使伸缩振动频率降低。

\subsubsection{溶剂的类型}
溶剂的类型:极性溶剂,吸收峰向低频区移动,吸收峰强度增大。

\subsubsection{物质的状态}
物质的状体:气态、液态、固态,吸收峰频率依次降低。