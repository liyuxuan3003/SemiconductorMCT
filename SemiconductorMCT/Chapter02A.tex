\section{质谱法的概述}

\uwave{质谱法分析法}(Mass Spectrometry, MS)是在高真空系统中测定样品的分子离子及碎片离子质量,以确定样品相对分子质量及分子结构的方法。化合物受到电子流冲击后,形成的带正电荷分子离子及碎片离子,按照其质荷比$m/z$大小依次排列而被记录下来的图谱,称为质谱。

质谱法是分子质量精确测定与化合物结构分析的重要工具。

\subsection{质谱法的发展历史}
质谱的简要发展历史如下
\begin{enumerate}
    \item 1912年,第一台单聚焦质谱仪。
    \item 1940年,第一台双聚焦质谱仪。
    \item 1955年,飞行时间质谱仪。
    \item 1960年,气相色谱--质朴(GC/MS)。
\end{enumerate}

\subsection{质谱法的特点}
质谱具有以下优点
\begin{enumerate}
    \item 应用范围广:测定样品,可以是无机物或有机物,可以是气体、液体、固体。
    \item 灵敏度较高:有机质谱仪的灵敏度可以达到$\SI{50}{pg}$,无机质谱仪的灵敏度可以达到$\SI{0.01}{pg}$。
    \item 样品用量少:质谱只需要微克级样品就可以得到满意的分析结果。
    \item 分析速度快:质谱不仅分析速度快,还可以实现多组分同时测定。
\end{enumerate}
质谱的主要问题是,仪器结构复杂,价格昂贵,维护困难,且测试对样品有破坏性。