\section{核磁共振氢谱}

\subsection{化学位移}
依照\fancyref{fml:核磁共振的共振吸收条件}中讨论的基本原理,氢核\cx{^1H}在某一特定的照射频率$\nu$下,只能在某一磁感应强度下发生核磁共振,换言之,所有氢核\cx{^1H}的共振磁场都相同?这并不正确!如果真的如此,核磁共振就没有任何意义了,所有物质都将得到相同结构。

实际情况是,尽管氢核\cx{^1H}的共振磁场都大致相同,但是处于不同化学环境的氢核\cx{^1H}的共振磁场会略有些不同!这微小的差异,就可以帮助我们推断待测物质中有哪些化学环境的氢,进而推定待测物质的结构。那么,为什么化学环境会影响共振磁场呢?这是\uwave{电子屏蔽效应}的结果,原子核的核外电子在绕核旋转的同时会产生与外磁场方向相反的感生磁场,因此,原子核感受到的有效磁感强度其实是外磁感强度$B_0$减去感生磁感强度$\sigma B_0$,$\sigma$被称为\uwave{屏蔽常数}。

\begin{BoxFormula}[核磁共振的电子屏蔽效应]
    核磁共振的共振吸收条件,在考虑电子屏蔽效应后,应修正为
    \begin{Equation}
        \nu=\frac{1}{2\pi}\gamma(1-\sigma)B_0=\nu_0
    \end{Equation}
\end{BoxFormula}

关于屏蔽效应,记住以下关系
\begin{itemize}
    \item \textbf{屏蔽效应越弱},$\sigma$越小,$(1-\sigma)$越大,扫场时$B_0$减小,扫频时$\nu$增大,\textbf{低场高频}。
    \item \textbf{屏蔽效应越强},$\sigma$越大,$(1-\sigma)$越小,扫场时$B_0$增大,扫频时$\nu$减小,\textbf{高场低频}。
\end{itemize}
因此,要考察氢的\uwave{化学环境},其实就是要通过核磁共振扫场测得的$B_0$或扫频测得的$\nu$推定屏蔽常数$\sigma$的值。但有一些问题,首先,扫场谱图以$B_0$绘制,扫频谱图以$\nu$绘制,那么两者的谱图是完全无法对照的。其次,即便是同一种扫描方式,例如,都是扫场,都以$B_0$为横坐标绘制谱图,若扫场采用的固定频率$\nu$不同,那么谱图也无法对照。综上,我们需要一种相对的度量。这就是\uwave{化学位移},记为$\delta$,它以某一标准物质的共振吸收峰$B_\te{标}$或$\nu_\te{标}$,测定样品种各共振吸收峰$B_\te{样}$或$\nu_\te{标}$,将其和标样的差值与标样作比,得到的无量纲数,就是化学位移。

\begin{BoxDefinition}[扫频时的化学位移]
    扫频时,化学位移$\delta$定义如下
    \begin{Equation}
        \delta=\frac{\nu_\te{样}-\nu_\te{标}}{\nu_\te{标}}\times 10^{6}=+\frac{\delt{\nu}}{\nu_\te{标}}\times 10^6\quad (\si{ppm})
    \end{Equation}
\end{BoxDefinition}

\begin{BoxDefinition}[扫场时的化学位移]*
    扫场时,化学位移$\delta$定义如下
    \begin{Equation}
        \delta=\frac{B_\te{标}-B_\te{样}}{B_\te{标}}\times 10^{6}=-\frac{\delt{B}}{B_\te{标}}\times 10^{6}\quad (\si{ppm})
    \end{Equation}
\end{BoxDefinition}
这里使用的单位$\si{ppm}$是百万分之一,这是因为样品和标准品的差距很小,故结果需乘$10^6$。

这里如果我们很仔细的话,很快就会发现一个严重的问题,化学位移$\delta$本质上是关于屏蔽系数$\sigma$的函数$\delta(\sigma)$,然而,按照两种方式定义的化学位移$\delta$最终得到的$\delta(\sigma)$其实是不一样的!

根据\fancyref{def:扫频时的化学位移},考虑到在磁场固定时$\nu\propto(1-\sigma)$
\begin{Equation}
    \delta_{\nu}=\frac{\nu_\te{样}-\nu_\te{标}}{\nu_\te{标}}=\frac{(1-\sigma_\te{样})-(1-\sigma_\te{标})}{(1-\sigma_\te{标})}=\frac{\sigma_\te{样}-\sigma_\te{标}}{\sigma_\te{标}-1}
\end{Equation}

根据\fancyref{def:扫场时的化学位移},考虑到在频率固定时$B\propto(1-\sigma)^{-1}$
\begin{Equation}
    \delta_B=\frac{B_\te{样}-B_\te{标}}{B_\te{标}}=\frac{(1-\sigma_\te{样})^{-1}-(1-\sigma_\te{标})^{-1}}{(1-\sigma_\te{标})^{-1}}=\frac{\sigma_\te{样}-\sigma_\te{标}}{\sigma_\te{样}-1}
\end{Equation}
注意到两者的分母确实不一样,这一点也可以在\xref{fig:化学位移和屏蔽常数的关系}中看出。
\begin{Figure}[化学位移和屏蔽常数的关系]
    \includegraphics{build/Chapter01B_01a.fig.pdf}
\end{Figure}
然而,这个问题不是很要紧,尽管$\sigma$具有$0$至$1$的变化范围,但是$\sigma_\te{样}$相对$\sigma_\te{标}$的偏差是非常非常小的,试想,化学位移的值是在$\si{ppm}$量级的\footnote{请注意图中纵轴并不是$\si{ppm}$单位的,纵轴的$-2$和$1$就相当于是$\delta=-2\times 10^6$和$\delta=1\times 10^6$,而实际的$\delta$只有个位数。}!可见变化范围有多么小。另外,\xref{fig:化学位移和屏蔽常数的关系}绘制时取$\sigma_\te{标}=0.4$也是严重偏离实际的,通常来说,所有物质的$\sigma$都是非常非常接近零的。


实践中,通常选用四甲基硅烷\cx{(CH3)4Si}作为标准品,简称TMS,其有以下几个优点
\begin{enumerate}
    \item TMS化学性质不活泼,与样品之间不发生化学反应和分子间缔合。
    \item TMS四个甲基化学环境完全相同,因此,无论在氢谱还是碳谱中都只有一个吸收峰。
    \item TMS相比一般化合物,屏蔽作用强,即$\sigma$比一般物质高,故其对于一般化合物的共振吸收不容易产生干扰。如\xref{fig:化学位移和屏蔽常数的关系}所示,屏蔽常数$\sigma$越小,化学位移$\delta$就越大。既然TMS的$\sigma$比大部分物质都大,若以TMS为标准品,那大部分物质的化学位移$\delta$都将是正值。
    \item TMS的沸点很低,仅为$\SI{27}{\degreeCelsius}$,容易去除,有利于样品回收。
\end{enumerate}\goodbreak

\xref{fig:四甲基硅烷}给出了四甲基硅烷TMS的结构式。
\begin{Figure}[四甲基硅烷]
    \chemfig{Si(-CH_3)(-[2]CH_3)(-[4]H_3C)(-[6]CH_3)}
\end{Figure}
化学位移和屏蔽常数、磁场、频率等的关系如下
\begin{itemize}
    \item 化学位移$\delta$越大,屏蔽效应$\sigma$越弱,低场高频。
    \item 化学位移$\delta$越小,屏蔽效应$\sigma$越强,高场低频。
\end{itemize}
\begin{Figure}[化学位移的图像]
    \includegraphics[scale=1.3]{build/Chapter01B_02.fig.pdf}
\end{Figure}

\subsection{化学位移的影响因素}
化学位移取决于核外电子云的密度,因此有诸多因素都可以影响化学位移。

\subsubsection{碳的取代基电负性的影响}
简而言之,\empx{电负性越强,化学位移越大}。这里说的电负性对化学位移的影响,是指取代基的电负性对于取代基上的氢的影响,例如\cx{CH3-H}、\cx{CH3-I}、\cx{CH3-Br}、\cx{CH3-Cl}、\cx{CH3-F}上的取代基的电负性是逐渐加强的,相应的甲基上的氢的化学位移也就越大。另外,\empx{电负性的影响随距离减弱},例如\xref{fig:化学位移与电负性的关系}所示的碘乙烷,距碘更近的\cx{H_b}就比距碘更远的\cx{H_a}有更大的化学位移。
\begin{Figure}[化学位移与电负性的关系]
    \chemfig{C(-[2]H_a)(-[4]H_a)(-[6]H_a)-C(-[2]H_b)(-[6]H_b)-I}
\end{Figure}
应指出的是,取代基电负性的影响是短程的,通常只能维持3至4个碳。

\subsubsection{碳的饱和程度的影响}
碳碳单键为$\te{sp}^3$杂化,碳碳双键为$\te{sp}^2$杂化,很明显的是,双键相较单键,s轨道电子的成分从25\%增加至33\%,我们知道,s轨道电子意味着电子更靠近碳原子,距离氢原子更远,因而对氢原子的屏蔽减弱了,换言之,双键碳上的氢相较单键氢上的碳具有更大的化学位移。

然而,不饱和度越大化学位移就越大吗?事情并没有那么简单
\begin{itemize}
    \item 烷烃(乙烷)的氢化学位移在$\delta=0.86$左右。
    \item 烯烃(乙烯)的氢化学位移在$\delta=5.25$左右。
    \item 炔烃(乙炔)的氢化学位移在$\delta=1.80$左右,比烯氢低。
    \item 芳烃(苯环)的氢化学位移在$\delta=7.27$左右,比烯氢高。
\end{itemize}
这是因为,电子云是具有方向性,电子云的屏蔽效应也是各向异性的,这种影响比距离要更为显著。乙烯的氢位于去屏蔽区,乙炔的氢位于屏蔽区,故后者的化学位移要小于前者。苯环形式上也是由双键组成的,但是在高中化学我们就知道,苯环,或者更一般的说,任何单双键交替的结构,都会形成共轭体系,使其表现出不同于双键的性质。苯环的环状共轭体系产生的环电流,使苯环表面处于屏蔽区($\delta$较大),使苯环上下的区域处于去屏蔽区($\delta$较小),由于苯环的六个氢均在平面上,处于苯环的屏蔽区内,故苯环的氢的化学位移均显著高于烯烃的氢。

\subsubsection{氢键的影响}
简而言之,\empx{氢键的形成会使化学位移增大},例如\cx{-OH}、\cx{-NH2}、\cx{-COOH}都能形成氢键。

\subsubsection{氘代溶剂的使用}
不过尚有一个问题,测试时样品总是以溶液的形态存在的,溶剂中的氢也会出峰,并且,溶剂的量远大于样品,溶剂峰会因此掩盖样品峰。为此,可以使用氘取代溶剂中的氢,称为\uwave{氘代溶剂},氘的共振频率和氢差别很大,氢谱中不会出现氘的峰,这样,也就减少了溶剂的干扰。

\subsection{耦合常数}
有趣的是,核磁共振氢谱的每一条谱线并不是简单的单峰,而是具有一定裂分形状的,这就是自旋耦合的结果。\uwave{自旋耦合}是指,临近核核自旋产生的核磁矩间的相互干扰,\uwave{自旋裂分}则是指由于自旋耦合产生的峰裂分现象。具体而言,例如\xref{fig:氯乙烷}所示的氯乙烷\cx{CH3-CH2-Cl}中
\begin{itemize}
    \item \cx{CH3}上的\cx{H_a}的谱线将裂分为三个峰,强度为$1:2:1$。
    \item \cx{CH2Cl}上的\cx{H_b}的谱线将裂分为四个峰,强度为$1:3:3:1$。
\end{itemize}\goodbreak
这是怎么发生的?以\cx{H_a}的谱线裂分为三个峰为例,由于\cx{H_a}受到\cx{H_b}的自旋耦合作用,我们知道,氢原子的自旋有$\pm 1/2$两种可能,\cx{H_b}共有两个,组合起来,就有三种可能性
\begin{itemize}
    \item 当两个\cx{H_b}的自旋为$-1/2,-1/2$时,总自旋为$-1$。
    \item 当两个\cx{H_b}的自旋为$-1/2,+1/2$或$+1/2,-1/2$时,总自旋为$0$。
    \item 当两个\cx{H_b}的自旋为$+1/2,+1/2$时,总自旋为$+1$。
\end{itemize}
这就是\cx{H_a}三根谱线和强度为$1:2:1$的来源,而\cx{H_b}四根谱线和强度$1:3:3:1$也可以类似得到解释,更一般的说,若临近的氢有$n$个,则总共有$n+1$根谱线,强度按杨辉三角分布。

\begin{Figure}[自旋裂分的示例]
    \begin{FigureSub}[氯乙烷]
        \chemfig{C(-[2]H_a)(-[4]H_a)(-[6]H_a)-C(-[2]H_b)(-[6]H_b)-Cl}
    \end{FigureSub}
    \qquad
    \begin{FigureSub}[二氟乙烯]
        \chemfig{C(-[4]H_a)(-[6]H_a)=C(-[6]F)-F}
    \end{FigureSub}
    \qquad
    \begin{FigureSub}[氟乙烯]
        \chemfig{C(-[4]H_a)(-[6]H_a)=C(-[6]H_b)-F}
    \end{FigureSub}
\end{Figure}
这里涉及到一个重要物理连,称为\uwave{耦合常数},记为$J_{ab}$,它被定义为一组峰内两个裂分的峰间的距离,它代表了两个核间的耦合作用的强弱,换言之,自旋耦合作用能多大程度的分裂两个峰?耦合常数的单位常用$\si{Hz}$表示,而不是化学位移。原因是,自旋耦合不同于化学环境的屏蔽作用,自旋耦合与核磁之间的相互作用是通过成键电子直接传递的,不涉及外磁场,这意味着,耦合常数$J_{ab}$是与外磁场无关的绝对量,不能用按照外磁场相对化的化学位移来度量。

这里还涉及到两个很重要的概念:\uwave{化学等价}和\uwave{磁等价}。化学等价就是我们很熟悉的化学环境相同,磁等价则是在一组原子化学等价的基础上,要求这组原子对分子内其他任何原子的自旋耦合的强度都相同,两者的关系可以描述为:\empx{磁等价一定化学等价},\empx{化学等价不一定磁等价}。
\begin{itemize}
    \item \xref{fig:氯乙烷}中的\cx{H_a}化学等价且磁等价,这是很显然的。
    \item \xref{fig:二氟乙烯}中的\cx{H_a}化学等价但为磁不等价,\xref{fig:氟乙烯}中的\cx{H_a}化学不等价且磁不等价,这是因为,\empx{碳碳单键可以自由转动},但是,\empx{碳碳双键不可以自由转动},\xref{fig:氟乙烯}的氟乙烯右侧的碳连接的两个基团不同,故左侧两个氢肯定是化学不等价的了,那自然也不可能是磁等价的了。\xref{fig:二氟乙烯}的二氟乙烯是对称的,因此氢是化学等价的,然而,仍然是由于双键不可旋转的问题,这两个氢到右侧(比如说,下侧的)氟的距离是不同的,因而耦合作用肯定是不同的,所以是化学不等价的。总之,麻烦的源头来自于碳碳双键不可旋转。
\end{itemize}

\subsection{峰面积}
氢谱不仅可以定性分析出有几种类型的氢,还可以定量给出它们的个数比例关系。这可以通过自动积分仪对峰面积进行自动积分,得到一个阶梯式的积分曲线,\empx{峰面积正比于氢的数目}。

\subsection{氢谱解析}
氢谱可以提供给我们的包含以下信息
\begin{itemize}
    \item 峰的数目:表示氢核的类型数。
    \item 峰的位移:表示(该类型)氢核所处的化学环境。
    \item 峰的裂分:表示(该类型)氢核相邻原子上的氢核数。
    \item 峰的强度:表示(该类型)氢核的相对数目。
\end{itemize}