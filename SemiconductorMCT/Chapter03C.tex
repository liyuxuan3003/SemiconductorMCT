\section{红外吸收光谱的仪器构造}
红外光谱的仪器构成:光源、单色器、样品室、探测器组成。

红外光谱的光源:应当能发射出高强度的连续红外光
\begin{itemize}
    \item 能斯特灯
    \item 碘钙灯
    \item 硅碳灯
    \item 炽热镍铬丝圈
    \item 高压汞灯
\end{itemize}
红外光谱的单色器:应当能机械的从较宽波长范围的光中选择性透射较窄的波长范围的光。

红外光谱的制洋常采用透射法(溴化碘压片法):
\begin{enumerate}
    \item 取干燥的样品(约$\SI{1}{mg}$)与干燥的\cx{KBr}(约$\SI{200}{mg}$)。
    \item 将样品和\cx{Kbr}在玛瑙研体中研磨至少一分钟,充分研细并混合均匀。
    \item 将混合后的试样居中放置到压片磨具间。
    \item 将磨具放置到压片机中,在$\SI{15}{MPa}$左右的压力下维持$\SIrange{30}{60}{s}$,得到透明的绽片。
\end{enumerate}
选择\cx{KBr}作为基料而不是\cx{NaCl}、\cx{KCl}、\cx{KF}的原因是,\cx{KBr}具有良好的透光性且自身吸收若弱,同时,\cx{KBr}稳定性好不容易受外界影响。\cx{NaCl}和\cx{KCl}自身有吸收,\cx{KF}则容易吸水。