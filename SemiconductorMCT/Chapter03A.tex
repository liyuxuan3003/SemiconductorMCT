\section{红外吸收光谱的基本原理}

\subsection{光谱的概念}
光是一种电磁波,具有波粒二象性,可以用频率$\nu$和波数$\hat{\nu}=1/\lambda$来描述
\begin{Equation}
    \nu=c\hat{\nu}
\end{Equation}
光量子的能量正比于其频率
\begin{Equation}
    E=h\nu
\end{Equation}
在分子光谱中,根据电磁波的波长$\lambda$可以划分出若干区域
\begin{Tablex}[分子光谱]{XXX}
<行为&电磁波类型&波谱\\>
电子自旋&射频区&--\\
原子核自旋&无线电波&核磁共振NMR\\
转动跃迁&微波&--\\
振动跃迁&红外&红外吸收光谱IR\\
电子跃迁&紫外--可见&紫外可见吸收光谱UV\\
化学键断裂&X射线&--\\
\end{Tablex}
请注意,质谱并不在其中,因为质谱不是一种光谱,它是一种质量谱。

\subsection{红外光的波谱范围}
红外光在光谱中,处于可见区域与微波之间。

红外光的波长范围是$\SIrange{0.78}{500}{um}$,波数范围是$\SIrange{12820}{20}{cm^{-1}}$。

红外光可以进一步依据其波长或波数,分为三个区域,如\xref{tab:红外光的波谱范围}所示

\begin{Tablex}[红外光的波谱范围]{lllXl}
    <区域&区域别称&波长($\si{um}$)&波数($\si{cm^{-1}}$)&能级跃迁类型\\>
    近红外区&泛频区&\SIrange{0.78e0}{2.5e0}{}&\SIrange{1.282e4}{4e3}{}&\cx{OH},\cx{NH},\cx{CH}的倍频吸收\\
    中红外区&基本振动区&\SIrange{2.5e0}{2.5e1}{}&\SIrange{4e3}{4e2}{}&振动、伴随转动\\
    远红外区&骨架振动区&\SIrange{2.5e1}{5.0e2}{}&\SIrange{4e2}{2e1}{}&振动、伴随转动\\
\end{Tablex}
原子和分子具有的能量是量子化的,称为原子或分子能级,包含
\begin{itemize}
    \item 电子能级
    \item 振动能级
    \item 转动能级
    \item 平动能级(光谱中不涉及)
\end{itemize}
原子和分子总的能量可以表达为三个部分的和
\begin{Equation}
    E=E_e+E_v+E_r
\end{Equation}
其中,$E_e,E_v,E_r$分别为电子能、振动能、转动能,依照\xref{tab:分子光谱}
\begin{itemize}
    \item 紫外光谱对应$E_e$,即电子能级。
    \item 红外光谱对应$E_v,E_r$,即振动能级和转动能级(转动能级其实部分属于微波)。
\end{itemize}
简而言之,\empx{红外光谱是由分子振动能级和转动能级产生的}。

\subsection{红外吸收光谱}
\uwave{红外吸收光谱}(Infrared Absorption Spectroscopy)是指,利用物质分子对红外光的吸收光谱来鉴别分子的组成和结构的方法。分子能选择性的吸收某些波长的红外线,引起分子振动能级和转动能级的跃迁,检测吸收情况即可得到吸收光谱。红外吸收光谱又称为\uwave{振动转动光谱}。
\begin{itemize}
    \item 红外吸收光谱的横坐标:波数$\hat{\nu}$,范围在$\SIrange{4e3}{4e2}{cm^{-1}}$,表示吸收峰位置。
    \item 红外吸收光谱的纵坐标:透过率$T\%$,表示吸收强度(峰是倒过来的)。
\end{itemize}
若记$I_0,I$分别为入射光和透射光的强度,则
\begin{Equation}
    T=\frac{I}{I_0}\times 100\%
\end{Equation}