\section{核磁共振概述}

\subsection{核磁共振的定义}
\uwave{核磁共振}(Nuclear Magnetic Resonance, NMR):在外加磁场中,具有自旋的原子核将发生自旋能级分裂,当吸收适当的电磁辐射后,核自旋能级发生跃迁,由此来测定物质结构。

\subsection{核磁共振的原理}
核磁共振的研究对象限于\uwave{磁性核},即具有磁矩的核,那什么样的核是磁性核呢?

我们都很熟悉电子自旋,类似的,原子核其实也是具有自旋的。我们知道,原子核由若干质子和中子组成,质子和中子均是由三个夸克构成,均具有$1/2$自旋。质子和中子组合构成原子核时,各自需要遵循泡利不相容,即必须一个自旋$+1/2$的质子搭配一个$-1/2$的质子,若总数是奇数则可以有一个落单的$\pm 1/2$的质子,中子同理。这就意味着,取决于质子和中子的数量
\begin{itemize}
    \item 质子和中子均为偶数,两者的自旋均为零,总自旋为$0$。
    \item 质子和中子均为奇数,两者各贡献$1/2$自旋,总自旋$1$(实际可以是整数)。
    \item 质子和中子有一奇数,两者中奇数的贡献$1/2$自旋,总自旋$1/2$(实际可以是半整数)。
\end{itemize}
上述在$1$和$1/2$外还可能取整数$1,2,3,\cdots$和半整数$1/2,\ 3/2,\ 5/2,\ \cdots $的原因是,原子核的自旋不仅来自质子和中子,还可能来自轨道自旋。这是比较复杂的,但所幸的是,核磁共振研究的主要都是$1/2$自旋的核,它们的电荷分布可以视为球体,共振吸收较简单。例如最常见的氢谱和碳谱核磁共振(为什么研究氢和碳?因为氢和碳是有机物的主要成分),就分别是针对\xce{^1H}和\xce{^{13}C}进行的,\xce{^1H}具有$1$个质子$0$个中子,\xce{^{13}C}具有$6$个质子和$7$个中子。两者都属于上面提到的质子数和中子数有一个是奇数的情况,都具有$1/2$自旋。当然这里可能会有一个问题,我们知道,\xce{^{13}C}作为碳同位素要比通常的\xce{^{12}C}丰度低的多,为什么选取\xce{^{13}C}进行核磁共振?因为\xce{^{12}C}是$6$个质子和$6$个中子,均为偶数,是非磁性核,故只能退而求其次。\goodbreak

自旋为$I$时磁量子数$m$可取$+I,+(I-1),+(I-2),\cdots,-(I-2),-(I-1),-I$,例如
\begin{itemize}
    \item 若自旋$I=1/2$,则$m=\pm 1/2$
    \item 若自旋$I=3/2$,则$m=\pm 1/2, \pm 3/2$
    \item 若自旋$I=1$,则$m=0, \pm 1$
    \item 若自旋$I=2$,则$m=0, \pm 1, \pm 2$
\end{itemize}
若核的磁量子数为$m$,则自旋角动量$P$为\setpeq{核磁共振原理}
\begin{Equation}&[1]
    P=\frac{h}{2\pi}m
\end{Equation}
磁矩$\mu$的表达式为,其中$\gamma$称为磁旋比
\begin{Equation}&[2]
    \mu=\gamma P
\end{Equation}
磁矩磁场相互作用能$E$为
\begin{Equation}&[3]
    E_0=-\mu B_0=-\gamma\mu B_0=-\gamma\frac{h}{2\pi}mB_0
\end{Equation}
而对于核磁共振研究的核,这里$m=\pm 1/2$
\begin{Equation}&[4]
    E_0(m=+1/2)=-\gamma\frac{h}{2\pi}(1/2)B_0\qquad
    E_0(m=-1/2)=+\gamma\frac{h}{2\pi}(1/2)B_0
\end{Equation}
两个自旋能级间的差值
\begin{Equation}&[5]
    \delt{E}_0=E(-1/2)-E(+1/2)=\frac{h}{2\pi}\gamma B_0
\end{Equation}
由此可见,自旋能级的间距正比于磁场$B_0$,换言之,磁场能使原先能量相同的自旋能级分裂。

而另外一方面,原子核可以通过电磁波的形式吸收能量,若电磁波的频率为$\nu$
\begin{Equation}&[6]
    \delt{E}=h\nu
\end{Equation}
而只有当电磁波的能量恰好等于自旋能级间的能量差时,电磁波才会被吸收,即
\begin{Equation}&[7]
    \delt{E}=\delt{E_0}
\end{Equation}
若约掉$h$,即得
\begin{Equation}&[8]
    \nu=\frac{1}{2\pi}\gamma B_0
\end{Equation}
这就是所谓的\uwave{共振吸收},当然我们好像至此只看懂了吸收,而没有看到共振?其实,\xrefpeq{8}右端的$\gamma B_0/2\pi$对于原子核有特殊含义,它代表原子核在磁场中的拉莫尔进动频率,记为$\nu_0$
\begin{Equation}
    \nu=\frac{1}{2\pi}\gamma B_0=\nu_0
\end{Equation}

至此,核磁共振就可以被解释为:在外加磁场下,电磁波与原子核进动的频率发生共振。\goodbreak

\begin{BoxFormula}[核磁共振的共振吸收条件]
    核磁共振的共振吸收条件为($\gamma$为原子核的磁旋比)
    \begin{Equation}
        \nu=\frac{1}{2\pi}\gamma B_0=\nu_0
    \end{Equation}
    其中,$B_0$为外加磁场,$\nu$为外加电磁波频率,$\nu_0$为外加磁场$B_0$下的原子核进动频率。
\end{BoxFormula}

而为了使核磁共振$\nu=\nu_0$发生(注意$\nu_0$完全关于外加磁场$B_0$),有两种方案
\begin{itemize}
    \item 第一种称为扫场(低场$\to$高场),固定电磁波频率$\nu$,变化外加磁场$B_0$
    \item 第二种称为扫频(低频$\to$高频),固定外加磁场$B_0$,变化电磁波频率$\nu$
\end{itemize}
通常来说,使用的较多的是扫场。特定的原子核在一定频率的电磁波$\nu$下,发生共振吸收的磁场$B_0$是一定的,这和原子核的$\gamma$有关。例如$\nu=60\si{MHz}$时共振吸收的磁场$B_0=1.4092\si{T}$。

\subsection{核磁共振的饱和弛豫}
核磁共振事实上并没有那么简单,因为,当一个核从低能态吸收频率$\nu=\nu_0$的电磁波跃迁至高能态后,实际上也有相同概率放出频率$\nu=\nu_0$的电磁波回到低能态。这就意味着,如果低能态核高能态的原子核数量相等,吸收和发射将抵消,我们就将无法观察到任何共振吸收现象了。最初低能态的原子核的数目是比高能态的原子核的数目多的,然而,随着核磁共振的进行,两者的数目会相同,这就是核磁共振的\uwave{饱和}。不过,所幸的是,饱和实际并不会出现,这是因为核磁共振还有\uwave{弛豫}。弛豫是指:\empx{通过非辐射方式,从高能态变为低能态}。弛豫的存在指出从高能态回到低能态也可以不发射电磁波,因此,即便两种能态数量相等,吸收仍然强于发射。

核磁共振种的弛豫分为两类
\begin{itemize}
    \item \uwave{自旋--晶格弛豫},也称为\uwave{纵向弛豫},半衰期记为$T_1$,是指高能态自旋核将能量传递给周围晶格或溶剂。固体的$T_1$需几小时,液体的$T_1$仅需几秒。故核磁共振多采用液体样品。
    \item \uwave{自旋--自旋弛豫},也称为\uwave{横向弛豫},半衰器记为$T_2$,是指高能态自旋核将能量传递给临近低能态自旋核。固体的$T_2$通常在$10^{-4}\si{s}$至$10^{-5}\si{s}$间,液体的$T_2$则通常在$1\si{s}$左右。
\end{itemize}

\subsection{核磁共振仪}
核磁共振仪可以按照以下几个维度进行分类
\begin{itemize}
    \item 按磁体:永磁体、电磁体、超导磁体。
    \item 按频率:\SIlist{60;80;90;100;200;300;400;600}{MHz}。
    \item 按射频源:连续波波谱仪(CW-NMR)、脉冲傅里叶变换波谱以(PFT-NMR)。
\end{itemize}
核磁共振仪的组成:磁铁探头、射频发生器、射频接收器、扫描发生器、信号放大及记录仪。