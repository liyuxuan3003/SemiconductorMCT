\section{紫外--可见吸收光谱的概念}

\subsection{紫外--可见吸收光谱}
\uwave{紫外--可见吸收光谱}(UV-visible Absorption Spectroscopy)的产生过程是,运动的分子外层电子吸收外来的能量(光能、电能、热能),产生电子能级跃迁,产生分子吸收光谱。

紫外--可见吸收光谱基于电子能级,红外吸收光谱基于振动能级和转动能级。

\section{紫外--可见吸收光谱的仪器}
紫外--可见吸收光谱的仪器构成:光源、单色器、样品室、检测器。

\subsection{光源}
光源用于产生一定波长范围的连续光,可用分为两类
\begin{itemize}
    \item 热辐射光源:用于可见光区,包含钨丝灯和卤钨灯,可用范围在$\SIrange{340}{2500}{nm}$。
    \item 气体放电光源:用于紫外光区,包含氢灯和氚灯,可用范围在$\SIrange{180}{375}{nm}$。
\end{itemize}

\subsection{单色器}
单色器能从光源辐射的复合光中分出单色光的光学装置,包含几个部分
\begin{itemize}
    \item 入射狭缝:光源的光由此进入单色器。
    \item 准光元件:透镜或凹面反射镜,使入射光成为平行光。
    \item 色散元件:棱镜或光栅,使复合光分解为单色光。
    \item 聚焦元件:透镜或凹面反射镜,使分光后单色光聚焦至出射狭缝。
    \item 出射狭缝:单色光由此离开单色器。
\end{itemize}

\subsection{样品室}
样品室用于盛放试样,可用分为两类
\begin{itemize}
    \item 玻璃吸收池,可以用于可见光区。
    \item 石英吸收池,可以用于可见光区和紫外光区。
\end{itemize}

\subsection{检测器}
检测器用于检测信号,测量单色光透过溶液后光强的变化,常用的检测器有
\begin{itemize}
    \item 光电池,光电池(硒光电池)的敏感范围是$\SIrange{300}{800}{nm}$,容易出现疲劳效应。
    \item 光电管,光电管在紫外--可见分光光度计上的应用最为广泛
    \item 光电倍增管,光电倍增管是检测微弱光最常用的光电元件,灵敏度比光电管高$200$倍。
\end{itemize}