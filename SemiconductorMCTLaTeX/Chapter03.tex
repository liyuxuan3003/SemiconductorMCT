\chapter{质谱法}

\section{质谱法}

\subsection{质谱法的概述}

质谱法的定义:质朴分析法(Mass Spectrometry, MS)是在高真空系统中测定样品的分子离子及碎片离子质量,以确定样品相对分子质量及分子结构的方法。

质谱法的原理:化合物分子受到电子流冲击后,形成的带正电荷分子离子及碎片离子,按照其质量$m$和电荷$z$的比值,即质荷比$m/z$大小依次排列而被记录下来的图谱,称为质谱。

质谱法的定位:分子质量精确测定与化合物结构分析的重要工具

\subsection{质谱法的历史}
\begin{itemize}
    \item 1912年,单聚焦质谱仪。
    \item 1940年,双聚焦质谱仪。
    \item 1955年,飞行时间质谱仪。
    \item 1960年,色谱--质谱连用。
    \item 1984年,电喷雾电离源(ESI)。
    \item 2002年,基质辅助激光解吸电离。
\end{itemize}

\subsection{质谱法的特点}
\begin{enumerate}
    \item 应用范围广,可以是无机物也可以是有机物。
    \item 灵敏度高,样品用量小。
    \item 分析速度快,多组分同时测定。
    \item 仪器结构复杂,价格昂贵,对样品具有破坏性。
\end{enumerate}

\subsection{质谱仪的构造组成}
\begin{enumerate}
    \item 进样系统(Sample Intoduction):气体扩散、直接进样、色谱进样。
    \item 离子源(Ionization Source):电子轰击、化学电离、场致电离、激光。
    \item 质量分析仪(Mass Filiter):单聚焦、双聚焦、飞行时间、四极杆。
    \item 检测器(Ion Detector)
\end{enumerate}

\subsubsection{进样系统}
进样系统的作用是高效重复的将样品引入到离子源中,并且不能造成真空度的降低。
\begin{itemize}
    \item 间歇式进样:对于气体及沸点不高易于挥发的样品。
    \item 直接探针进样:对于高沸点的液体和固体。
    \item 色谱进样:色谱质谱联用,进行多组分复杂混合物分析。
\end{itemize}

\subsubsection{离子源}

离子源的功能是将进样系统引入的气态样品分子转化为离子。

离子化所需要的能量随分子不同差异很大,因此,对于不同的分子应选择不同的离子化方法。
\begin{itemize}
    \item 硬电离:需要较大能量电离的样品
    \item 软电离:需要较小能量电离的样品(适用于易碎裂或易电离的样品)
\end{itemize}

电子轰击电离法是通用的电离法,是使用高能电子束从试样分子中撞出一个电子而产生正离子,当电子源具有足够的能量时,有机物的分子可能不仅只失去一个电子形成分子离子,而且有可能断裂化学键,形成大量各种低质量数的碎片离子。

离子源的类型
\begin{itemize}
    \item 电子轰击电离(Electron Impaction Ionization, EI)
    \item 化学电离(Chemical Ionization, CI)
    \item 场电离(Field Ionization, FI)
    \item 场解吸
\end{itemize}
电子轰击电离
\begin{itemize}
    \item 电子能量降低,分子离子增加。
    \item 电子能量增加,碎片离子增加。
\end{itemize}
电子轰击电离的优点
\begin{itemize}
    \item 结构简单,稳定,电离效率高,易于实现
    \item 质谱图再现性好,便于计算机检索及比较
    \item 离子碎片多,可提供较多的分子结构信息
\end{itemize}
电子轰击电离的缺点:当样品分子稳定性不高时,分子离子峰的强度低。

\subsubsection{化学电离}
离子室内的反应气(甲烷、异丁烷、氨)被电子轰击先产生离子,离子再与试样分子碰撞反应,转移一个质子给试样或由试样移去一个电子,试样则变成代一个电荷的离子。

化学电离的优点

\subsection{质谱仪的工作要求}
质谱仪需要在高真空下工作:质谱仪的离子源和质朴分析仪必须处于高真空状态。