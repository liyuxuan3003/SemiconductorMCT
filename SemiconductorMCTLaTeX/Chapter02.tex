\chapter{核磁共振}

% \section{概述}

% \section{核磁共振氢谱}

% % 核磁共振氢谱可提供的重要信息
% % \begin{enumerate}
% %     \item 化学位移
% %     \item 耦合常数
% %     \item 峰的裂分情况
% %     \item 峰面积
% % \end{enumerate}

% % \subsection{化学位移}
% % 根据前面讨论的基本原理,\xce{^1H}自旋核在某一照射排列,只能在某一磁感应强度下发生核磁共振,但事实证明,\xce{^1H}在分子中所处的化学环境(例如\xce{-OH}、\xce{=CH2}、\xce{-CH3}对于\xce{^1H}就是三个不同的化学环境)不同时,即使在相同照射频率下,也将在不同的共振磁场下显示吸收峰。

% % \subsubsection{电子屏蔽效应}
% % 电子屏蔽效应:带正电原子核的核外电子在与外磁场垂直的平面上绕核旋转的同时,会产生与外磁场方向相反的感生磁场。所以质子实际上感受到的有效磁感应强度应是外磁场强度减去感生磁场强度。感生磁场的大小用$\sigma B_0$表示,$\sigma$为屏蔽常数,与核外电子云的密度有关。
% % \begin{Gather}[6pt]
% %     B_\te{有效}=(1-\sigma)B_0\\
% %     \nu_\te{有效}=\frac{\gamma}{2\pi}(1-\sigma)B_0
% % \end{Gather}
% % \begin{itemize}
% %     \item 屏蔽效应增强,共振信号移向高场(低频)。
% %     \item 屏蔽效应减弱,共振信号移向低场(高频)。
% % \end{itemize}

% % \subsubsection{化学位移}
% % 化学位移:不同化学环境的\xce{^1H}核受到的不同程度的屏蔽效应的影响,因而吸收峰出现在核磁共振谱不同位置上,吸收峰这种位置上的差异称为化学位移,用$\delta$表示。

% % \subsubsection{化学位移的表示}
% % 化学位移的相对值表示:以某一标准物质的共振吸收峰$v_\te{标}$或$B_\te{标}$为标准,测出各样品中各共振吸收峰与标准样品的的差值$\delt{v}$或$\delt{B}$,进一步除以$v_\te{标}$或$B_\te{标}$得到的无量纲量$\delta$。
% % \begin{Equation}
% %     \delta=\frac{\nu_\te{样}-\nu_\te{标}}{\nu_\te{标}}\times 10^6\si{ppm}\qquad
% %     \delta=\frac{B_\te{标}-B_\te{样}}{B_\te{标}}\times 10^6\si{ppm}
% % \end{Equation}

% % $\delta$值与所处化学环境有关,故称为化学位移。

% % $\delta$的单位采用\si{ppm}(百万分之一),因为差值很小。

% % \subsubsection{标准物}
% % 在化学位移测定时,常用沟道标准物是四甲基烷烃\xce{(CH3)4Si},简称TMS。

% % TMS用作标准物的优点是
% % \begin{enumerate}
% %     \item TMS的化学性质不活泼,与样品之间不发生化学反应和分子间缔合。
% %     \item TMS的四个甲基有相同的化学环境,因此无论在氢谱还是在碳谱中都只有一个吸收峰。
% %     \item TMS相比一般化合物核外电子屏蔽效应作用较大,信号位于高场端,对一般化合物的吸收不产生干扰。
% %     \item TMS的沸点很低($27\si{\degreeCelsius}$),容易去除,有利于回收样品。
% % \end{enumerate}

% % 若为扫场(频率$\nu$固定)
% % \begin{Equation}
% %     \delt{E}=h\nu=\frac{h}{2\pi}rB(1-\sigma)
% % \end{Equation}
% % \begin{itemize}
% %     \item $\sigma$越大,磁场越大,信号出现在谱图的右端(高场端)。
% %     \item $\sigma$越小,磁场越小,信号出现在谱图的左端(低场端)。
% % \end{itemize}

% % 若为扫频率(磁场$B_0$固定)
% % \begin{Equation}
% %     \nu=\frac{\gamma}{2\pi}(1-\sigma)B_0
% % \end{Equation}
% % \begin{itemize}
% %     \item $\sigma$越大,频率越小,信号出现在谱图的右端(低频端)。
% %     \item $\sigma$越小,频率越大,信号出现在谱图的左端(高频端)。
% % \end{itemize}

% % 结论是
% % \begin{itemize}
% %     \item 核外电子云密度高,屏蔽作用大($\sigma$值大),核的共振吸收向高场低频移动,化学位移小。
% %     \item 核外电子云密度低,屏蔽作用小($\sigma$值小),核的共振吸收向低场高频移动,化学位移大。
% % \end{itemize}

% % \subsubsection{化学位移的影响因素}
% % 化学位移取决于核外电子云密度,因此影响电子云密度的各种因素都对化学位移有影响,影响最大的是电负性和各向异性
% % \begin{enumerate}
% %     \item 取代基电负性越强,越移向低场。
% %     \item 相连碳原子的sp杂化轨道的s成分越多,越移向低场(共轭效应、共振效应)。
% %     \item 环状共轭体系的环电流效应。
% %     \item 磁各向异性。
% %     \item 氢键,氢键越强,越移向低场。
% %     \item 介质的影响。
% %     \item 范德华力的影响。
% % \end{enumerate}

% % \subsubsection{电负性}
% % 电负性增加,电子密度减小,化学位移$\delta$增加
% % \begin{Table}[基团的电负性]
% % <基团&电负性&$\delta$\\>
% %     TMS&1.8&0.0\\
% %     \ce{CH3-H}&2.1&0.2\\
% %     \ce{CH3-I}&2.5&2.16\\
% %     \ce{CH3-Br}&2.8&2.68\\
% %     \ce{CH3-Cl}&3.1&3.05\\
% %     \ce{CH3-OH}&3.5&3.4\\
% %     \ce{CH3-F}&4.0&4.26\\
% % \end{Table}
% % 取代电负性带来的屏蔽效应是短程的($<$3-4个碳)。

% % 相连碳原子的sp杂化:与氢相连的碳原子从sp3到sp2,s电子的成分从25\%增加到33\%,键电子更靠近碳原子,因而对相连的氢原子云有去屏蔽作用,即共振位置移向低场。至于炔氢谱峰相对烯氢处于较高场,芳环氢谱峰相对于烯氢处于较低场,则是另有较重要的影响因素所致。

% % \paragraph{双键和羰基}
% % 乙烯的质子$\delta$($\delta=5.25\si{ppm}$)比饱和烃的\xce{CH2}的$\delta$约大$4\si{ppm}$。碳氢双键的去屏蔽作用和碳碳双键相似,醛共振峰出现在低场($\delta=9.2\si{ppm}$)。

% % \paragraph{三键}
% % 乙炔,即碳碳三键是直线构型,$\pi$电子云围绕碳碳三键呈筒形分布,形成环电流,它所产生的感生磁场与外加磁场方向相反,故三键上的一个氢质子处于屏蔽区。屏蔽效应较强。

% % \paragraph{氢键}
% % 形成的氢键后,核外电子云密度降低,屏蔽效应减弱,$\delta$增加。

% % \paragraph{环状共轭体系的环电流效应}
% % 苯环的上下方(正屏蔽区):与$B_0$相反。

% % 苯环的外侧(去屏蔽区):与$B_0$同向。

% % \subsubsection{氚代溶剂}
% % 因为测试时溶剂中的氢也会出峰,溶剂的量远远大于样品中的量,溶剂峰会掩盖样品峰,所以用氘取代溶剂中的氢,氘的共振峰频和氢差别很大,减少了溶剂的干扰。

% % \subsection{耦合常数}

% % \paragraph{自旋耦合和自旋分裂}
% % \begin{enumerate}
% %     \item 自旋--自旋耦合:临近核核自旋产是的核磁矩间的相互干扰。
% %     \item  
% % \end{enumerate}

% % 耦合常数:每组吸收峰内各峰之间的距离,称为耦合常数,以$J_{ab}$表示,下标$a$和$b$表示相互耦合的磁不等性氢核的种类。

% % \begin{Equation}
% %     \nu_{a1}=\frac{\gamma}{2\pi}(1-\sigma)(B_0-\delt{B_b})\qquad
% %     \nu_{a2}=\frac{\gamma}{2\pi}(1-\sigma)(B_0+\delt{B_b})
% % \end{Equation}

% % \begin{Equation}
% %     J_{ab}=\nu_{a2}-\nu_{a1}=\frac{\gamma}{\pi}(1-\sigma)\delt{B_b}
% % \end{Equation}

% % \begin{itemize}
% %     \item $J$大小反应核间耦合作用强弱。
% %     \item $J$与外加磁场大小$(B_0)$无关。
% % \end{itemize}

% % 化学位移随外磁场的改变而改变,耦合常数与化学位移不同,它不随外磁场的改变而改变,因为自旋耦合产生的核磁之间的相互作用是通过成键电子来传递的,并不涉及磁场。

% % \subsubsection{耦合常数与自旋分裂的规律}
% % \begin{enumerate}
% %     \item 等价质子尽管有耦合,但没有分裂的规律,信号仍为单峰。
% %     \item 质子NMR裂分的数目,普遍公式是$2nI+1$,式中$I$是核自旋量子数。对于氢核,$I=1/2$,故公式简化为$n+1$。
% %     \item 自旋核的化学位移$\delta$位于各组分裂峰的中心点。
% %     \item 裂分峰的强度之比遵循二项式$(a+b)^n$的展开式的系数之比。
% % \end{enumerate}

% % \subsubsection{化学等价}

% % 在核磁共振中,有相同化学环境核具有相同的化学位移,那么这种核称为化学等价核。

% % \subsubsection{磁等价}

% % 分子中一组化学等价核与分子中的其他任何一个核具有相同强弱的耦合,则这组核称为磁等价核
% % \begin{itemize}
% %     \item 组内核的化学位移相等。
% %     \item 组外核耦合常数时,耦合常数相等。
% %     \item 在无组外核干扰时,组内虽耦合,但不裂分。
% % \end{itemize}
% % 磁等价核必定为化学等价核,化学等价并不一定为磁等价。

% % \subsection{峰面积}
% % 积分曲线核积分面积:决定氢质子数的办法

% % 一个化合物有几组吸收峰,取决于分子中\xce{H}核的化学环境,有几种不同类型的\xce{H}核,就有几个不同的吸收峰。



% \section{核磁共振碳谱}

% % \chemfig{C(-[3]H)(-[5]H)=C(-[1]H)(-[7]H)}

